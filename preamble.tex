% The file intend to keep track of good practices in Latex writing.

%==============================
% DOCUMENT
%==============================

% Fix some error reporting
\vfuzz2pt % Don't report over-full v-boxes if over-edge is small
\hfuzz2pt % Don't report over-full h-boxes if over-edge is small

% All the same, there are commands, classes and packages which are outdated and superseded. 
% nag provides routines to warn the user about the use of those.
\usepackage[l2tabu,orthodox]{nag}

%==============================
% BIBLIOGRAPHY
%==============================

% \bibliography{filename.bib} % in your preamble
% \citet{key}, \citep{key} % in the document
% \printbibliography % to generate the reference section
\usepackage[
backend=bibtex8, 
style=ieee, 
sorting=none, 
natbib=true, 
doi=false, 
isbn=false, 
url=false, 
eprint=false, 
maxcitenames=1, 
mincitenames=1
]{biblatex}

%==============================
% TEXT
%==============================

% \autoref{key} % instead of Figure~\ref{key}, Table~\ref{key}, or Section~\ref{key}
\usepackage[pdftex,colorlinks]{hyperref}
\def\sectionautorefname{Section}
\def\subsectionautorefname{Section}

% \acrodef{ICP}{Iterative Closest Point} % in the preamble
% \ac{ICP} % in the document
\usepackage{acronym}

%==============================
% FIGURE
%==============================

% Preferred figure format:
% - pdf or eps for graphs and schemas
% - jpg for photo

% \includegraphics[width=\textwidth]{filename}
\usepackage[pdftex]{graphicx}

% convert eps to pdf, you need to skip the file extension to work properly
% \includegraphics{filename} % instead of \includegraphics{filename.eps}
\usepackage{epstopdf}

\graphicspath{{./latexGoodPractices/}}

%==============================
% TABLE
%==============================

% \toprule, \midrule, \bottomrule % instead of \hline
\usepackage{booktabs}

%==============================
% MATH
%==============================

% Better symboles
\usepackage{amssymb,amsfonts,amsmath,amscd}

% \bm % in equations for proper bold font
\usepackage{bm} 

% Some handy commands
\newcommand{\norm}[1]{\left\Vert#1\right\Vert}
\newcommand{\abs}[1]{\left\vert#1\right\vert}
\newcommand{\set}[1]{\left\{#1\right\}}
\newcommand{\Real}{\mathbb R}
\newcommand{\bbm}{\begin{bmatrix}}
\newcommand{\ebm}{\end{bmatrix}}


