\newcommand\hmmax{0}
\newcommand\bmmax{0}

\PassOptionsToPackage{hyphens}{url}\usepackage[hidelinks]{hyperref}

% <<<<<<<<<<<-Stuff from preamble icra
% For checkmark and X mark
\usepackage{pifont}
\newcommand{\cmark}{\ding{51}}% Checkmark
\newcommand{\xmark}{\ding{55}}% X mark
\usepackage{supertabular}
\usepackage{rotating,tabularx}
% end of preamble icra -->>>>>>>>

% <<<<<<<<<<<- Stuff from preamble fsr
\usepackage{pdflscape}      % for \landscape to rotate the page in pdf
%\usepackage{mathptmx}       % selects Times Roman as basic font    -> Make your thesis look awful! :(
\usepackage{helvet}         % selects Helvetica as sans-serif font
\usepackage{courier}        % selects Courier as typewriter font
\usepackage{type1cm}        % activate if the above 3 fonts are
\usepackage{makeidx}         % allows index generation
\usepackage[binary-units]{siunitx}
\sisetup{group-separator = \text{\,}} % small space for thousand separator
%\usepackage{graphicx}        % standard LaTeX graphics tool
% when including figure files
\usepackage{multicol}        % used for the two-column index
\usepackage[bottom]{footmisc}% places footnotes at page bottom
\usepackage[normalem]{ulem}  % pour sout
% Add subfigure
\usepackage{subcaption}
% sidecaption
\usepackage{sidecap}
% Put text over photos
\usepackage[percent]{overpic}
% to put vertical caption
\usepackage{copyrightbox}
% Better symboles
%\usepackage{amsfonts,amsmath,amscd}
%Patch for a widecheck
%% code from mathabx.sty and mathabx.dcl
\DeclareFontFamily{U}{mathx}{\hyphenchar\font45}
\DeclareFontShape{U}{mathx}{m}{n}{
	<5> <6> <7> <8> <9> <10>
	<10.95> <12> <14.4> <17.28> <20.74> <24.88>
	mathx10
}{}
\DeclareSymbolFont{mathx}{U}{mathx}{m}{n}
\DeclareFontSubstitution{U}{mathx}{m}{n}
\DeclareMathAccent{\widecheck}{0}{mathx}{"71}
\DeclareMathAccent{\wideparen}{0}{mathx}{"75}
% Colored text
\usepackage{xcolor}
% end preambule from fsr -->>>>>>>>

%\newcommand{\philg}[1]{\textcolor{blue}{#1}}
%\newcommand{\babin}[1]{\textcolor{red}{#1}}

\usepackage[printonlyused, nohyperlinks]{acronym}
\usepackage{algorithm}
\usepackage{algpseudocode}
\usepackage{amsfonts}
\usepackage{amsmath} % called by mathtools
% \usepackage[toc,page]{appendix}
%\usepackage{bbm}

\bibliographystyle{unsrtnat}
\usepackage[
backend=bibtex, 
style=ieee, 
sorting=none, 
natbib=true, 
doi=false, 
isbn=false, 
url=false, 
eprint=false, 
maxcitenames=1, 
mincitenames=1]{biblatex}

\usepackage{bm}
\usepackage[font=small]{caption}
\usepackage{dirtytalk}
\usepackage{float}
\usepackage[T1]{fontenc}
\usepackage{graphicx}
\usepackage[utf8]{inputenc}
%\usepackage{mathabx}
%\usepackage{mathpazo} % Set the font to Palatino.
\usepackage{mathtools}
\usepackage{multirow}
%\usepackage[french,english]{babel}  % Babel is imported by ulthese...
\usepackage{pdfpages}
\usepackage{rotating}
\usepackage{siunitx}
\usepackage{tabu}
\usepackage{tabularx}
\usepackage[]{url}

\usepackage{threeparttablex} % For M-estimator appendice
\usepackage{longtable} % same

% For dates and time
\usepackage{datetime2}
\DTMnewtimestyle{custom}{
	\renewcommand{\DTMdisplaytime}[3]{
		\DTMtexorpdfstring{\DTMtwodigits{##1}:\DTMtwodigits{##2}}
	}
}
\DTMsettimestyle{custom}

\addto\extrasenglish{%
	\renewcommand{\chapterautorefname}{Chapter}%
	\renewcommand{\sectionautorefname}{Section}%
	\renewcommand{\subsectionautorefname}{Subsection}%
}

% For table arrows
\usepackage{mathabx}

% To autoref an appendix section
%\newcommand{\aref}[1]{\hyperref[#1]{Appendix~\ref{#1}}}

%% New column types for tabularx
%\newcolumntype{C}[1]{>{\centering\let\newline\\\arraybackslash\hspace{0pt}}m{#1}}
%\newcolumntype{Y}{>{\centering\arraybackslash}X}

% Rotated page have rotated page number
\usepackage{everypage}
\newcommand{\Lpagenumber}{\ifdim\textwidth=\linewidth\else\bgroup
	\dimendef\margin=0 %use \margin instead of \dimen0
	\ifodd\value{page}\margin=\oddsidemargin
	\else\margin=\evensidemargin
	\fi
	
	\thispagestyle{empty} 
	\raisebox{\dimexpr -\topmargin-\headheight-\headsep-0.5\linewidth}[0pt][0pt]{%
		\rlap{\hspace{\dimexpr \margin+\textheight+\footskip}%
			\llap{\rotatebox{90}{\thepage}}}}%
	\egroup\fi}
\AddEverypageHook{\Lpagenumber}%


%\defbibheading{subbibheading}{
	%  \section*{References (Chapter \ref{refsegment:\therefsection\therefsegment})}
	%}

% For tables / figures merged
\usepackage{graphicx,booktabs}

% For better tables
\usepackage{booktabs}

% Tables that can fit page length
\usepackage{tabularx}
\usepackage{multirow, multicol}

\usepackage{xspace} %smart handling of space in commands
\newcommand{\ie}{i.e.,\xspace{}}
\newcommand{\eg}{e.g.,\xspace{}}

\acrodef{SSMR}{skid-steer mobile robot}
\acrodef{ICR}{instantaneous center of rotation}
\acrodef{ROC}{radius of curvature}
\acrodef{GPR}{ground-penetrating radar}
\acrodef{EKF}{extended Kalman filter}
\acrodef{MI-UKF}{multi-innovation unscented Kalman filter}
\acrodef{4WD}{four-wheel drive}
\acrodef{ICP}{Iterative-Closest-Point}
\acrodef{UGV}{unmanned ground vehicle}
\acrodef{ROS}{Robot Operating System}
\acrodef{IMU}{Inertial Measurement Unit}
\acrodef{RMSRE}{root mean squared relative error}
\acrodef{SLAM}{simultaneous localization and mapping}
\acrodef{SOTA}{state-of-the-art}
\acrodef{SSMR}{skid-steering mobile robot}
\acrodef{AMR}{Ackermann mobile robot}
\acrodef{UGV}{uncrewed ground vehicle} %FP: more gender neutral
\acrodef{IDD}{ideal differential-drive}
\acrodef{ICR}{instantaneous center of rotation}
\acrodefplural{ICR}[ICRs]{instantaneous centers or rotation}
\acrodef{RTK}{Realtime Kinematics}
\acrodef{GNSS}{Global Navigation Satellite System}
\acrodef{ROC}{radius of curvature}
\acrodef{IMU}{inertial measurement unit}
\acrodef{MPC}{model predictive control}
\acrodef{GP}{Gaussian process}
\acrodefplural{GP}[GPs]{Gaussian processes}
\acrodef{BLR}{Bayesian linear regression}
\acrodef{IPEM}{integrated prediction error minimization}
\acrodef{MLP}{multilayer perceptron}
\acrodef{MRMSE}{multi-step root mean squared error}
\acrodef{T-MRMSE}{translational multi-step root mean squared error}
\acrodef{R-MRMSE}{rotational multi-step root mean squared error}
\acrodef{M-Z-score}{multi-step Z-score}
\acrodef{DRIVE}{Data-driven Robot Input Vector Exploration}
\acrodef{SNOW}{Self-driving Navigation Optimized for Winter}
\acrodef{AMSL}{Above Mean Sea Level}
\acrodef{DoF}{degree of freedom}
\acrodef{COM}{center of mass}
\acrodef{VTR}[VT\&R]{Visual Teach and Repeat}
\acrodef{LTR}[LT\&R]{Lidar Teach and Repeat}
\acrodef{CNN}{Convolutional Neural Network}
\acrodef{GeRoNa}{Generic Robot Navigation}
\acrodef{OE}{Orthogonal-Exponential}
\acrodef{DD-ORTHEXP}{Differential-Drive-ORTHEXP}
\acrodef{POI}{Points of Interest}
\acrodef{CAD}{Computer-Aided Drawing}
\acrodef{GPS}{Global Positioning System}
\acrodef{RAM}{Random Access Memory}
\acrodef{WILN}[WILN]{Weather-Invariant Lidar-based Navigation}

\newcommand{\DOUGHNUTCALIB}{\ac{DRIVE}\xspace}
\newcommand{\ICRBASED}{\ac{ICR}-based\xspace} 
\newcommand{\DIMPARAMS}{\bm k}
\newcommand{\HORDUR}{h_d}
\newcommand{\WINSIZE}{h}
\newcommand{\TRAINDATA}{$\mathcal{D}$\xspace}
\newcommand{\DOUGHNUTDATA}{$\mathcal{D}_D$}
\newcommand{\HUMANDATA}{$\mathcal{D}_H$}
\newcommand{\PREDSTATE}{\,^\mapf\!\hat{\bm q}}
\newcommand{\MEASSTATE}{\,^\mapf\bm q}
\newcommand{\MARMOTTE}{HD2\xspace}
\newcommand{\CMDBODYVEL}{^{\robotf}\bm{f}}
\newcommand{\SLIPBODYVEL}{^{\robotf}\bm{s}}
\newcommand{\OBSERVEDSLIP}{\bm{s}}
\newcommand{\INPUTVECTOR}{\bm{u}}
\newcommand{\STATEPROPMAT}{_{\robotf}^{\mapf}\bm{T}\left(^\mapf\theta_t\right)}
\newcommand{\INPUTSPACE}{\mathcal{J}}
\newcommand{\BODYVELSPACE}{\mathcal{B}}
\newcommand{\CENTRIFUGAL}{\psi}
\newcommand{\OPTIMCALIBTIME}{$t_{\text{opt}}$\xspace}
\newcommand{\robotstate}{\bm{q}}
\newcommand{\mapf}{\mathcal{G}} % map reference frame
\newcommand{\robotf}{\mathcal{R}} % robot reference frame
\newcommand{\learnedslip}{\bm{\Xi}}
\newcommand{\blrweights}{\bm{\gamma}}
\newcommand{\blrinputs}{\bm{x}}
\newcommand{\MRSME}{\epsilon}
\newcommand{\SENSORMEASUREMENTS}{\bm{z}}
\newcommand{\NITRO}{\ac{WILN}\xspace} 



%For tikz figure
\usepackage{tikz}
\usepackage{pgfplots}
\usepackage{csvsimple}
\usetikzlibrary{shapes}




% Text commands
\newcommand{\laverdiere}{\emph{Cabin}\xspace} % Laverdière
\newcommand{\quebec}{Qu\'{e}bec\xspace} % Québec
\newcommand{\foretmo}{For\^{e}t Montmorency\xspace} % Québec

% Useful math commands
\newcommand{\odomf}{\mathcal{O}} % odom reference frame
\newcommand{\lidarf}{\mathcal{L}} % lidar reference frame
\newcommand{\pathf}{\mathcal{S}} % path Frenet-Serret frame
\newcommand{\skreadpc}{\mathcal{P}_s} % Skewed reading point cloud
\newcommand{\readpc}{\mathcal{P}} % reading point cloud
\newcommand{\refpc}{\mathcal{J}} % reference point cloud
\newcommand{\refpoint}{\bm{j}} % reference point cloud
\newcommand{\match}{\mathcal{M}} % set of point matches
\newcommand{\weightset}{\mathcal{W}} % set of point weights
\newcommand{\transform}[2]{$_{#1}^{#2}\bm{T}$} % Transform from #1 frame to #2 frame
\newcommand{\reftraj}{\bm q_{\text{ref}}} % Reference trajectory
\newcommand{\poseplane}{\bm q_{\text{2D}}} % 2D robot pose
\newcommand{\cterr}{\epsilon_{CT}} % Cross-Track error
\newcommand{\curv}{\kappa} % Path curvature
\newcommand{\prior}[2]{${}_{#1}^{#2}\bm{\check{T}}$} % prior transform
\newcommand{\estimate}[2]{${}_{#1}^{#2}\bm{\hat{T}}$} % estimate transform
\newcommand{\kdtree}{kd-tree${}^5$\xspace} % kd tree
\newcommand{\robotroll}{^\mathcal{\mapf}\psi} % roll angle
\newcommand{\robotpitch}{^\mathcal{\mapf}\alpha} % pitch angle
\newcommand{\robotyaw}{^\mathcal{\mapf}\theta} % yaw angle

% icp params
\newcommand{\randsubratio}{\eta_s}
\newcommand{\bbox}{\bm b}
\newcommand{\matchernn}{n_m}
\newcommand{\matchermaxdist}{d_{\text{max}}}
\newcommand{\matcherepsilon}{\varepsilon}
\newcommand{\outlierratio}{\eta_d}
\newcommand{\checkererrortran}{\varepsilon_{{t}_{\text{min}}}}
\newcommand{\checkererrorrot}{\varepsilon_{{\theta}_{\text{min}}}}
\newcommand{\maxiticp}{i_\text{max}}
\newcommand{\maxdistappend}{\rho}
\newcommand{\normalsnn}{n_n}
\newcommand{\dynthresh}{\tau_d}
\newcommand{\voxelsize}{v_s}
\newcommand{\reftrajdist}{d_{\text{ref}}}

% PF params
\newcommand{\nomlinvel}{v_{\text{nom}}} % Nominal velocity
\newcommand{\maxlinvel}{v_{\text{max}}} % Max velocity
\newcommand{\minlinvel}{v_{\text{min}}} % Min velocity
\newcommand{\maxrotvel}{\omega_m}
\newcommand{\OEk}{\beta}
\newcommand{\OEKh}{\beta_h}
\newcommand{\OEKg}{\beta_g}
\newcommand{\goaltol}{\tau_{g}}
\newcommand{\waypointtol}{\tau_{w}}

% argmin and argmax
\DeclareMathOperator*{\argmin}{arg\,min}
\DeclareMathOperator*{\floor}{floor}
\DeclareMathOperator*{\ceil}{ceil}

% math operators for ICP
\DeclareMathOperator*{\error}{e}
\DeclareMathOperator*{\weight}{w}

% for mean
\newcommand*\mean[1]{\bar{#1}}

% Fill the template with text
\usepackage{lipsum}
% Better command to make sure that people don't confuse lipsum text with real text
\newcommand{\lightlipsum}[1][0]{\textcolor{gray!50}{\lipsum[#1]}}


\addbibresource{references-main.bib}
%\addbibresource{../references_crv2020.bib}
%\addbibresource{references_ICRA_2023.bib}
%\addbibresource{references_IROS_2023.bib}